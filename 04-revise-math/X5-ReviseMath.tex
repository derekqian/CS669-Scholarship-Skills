\documentclass[12pt]{article}
\textwidth=35pc
\textheight=54pc
\topmargin=-3pc
\linespread{1.3}  % gives about one-and-one-half line spacing
%\linespread{1.6}	% gives about double line spacing

\begin{document}
\begin{center}
\large
\bf
Scholarship Skills 2013\\[1ex]  % an ex is the height of a letter x, and thus depends on the current font size

\Large
Revise Mathematics
\normalsize

\vspace{1ex}
Assigned Thursday 31$^{st}$ January; due Thursday, 7$^{th}$ February 2013.
\\[2pc]  % pc is the abbreviation for a pica, 1/6 inch
\end{center}


\noindent
Apply what you've learned about writing mathematics to rewrite this proof.  
Don't be afraid to \emph{rewrite} it, rather than tinker about with it in small ways.
The \LaTeX{} source for this proof is on the web site, so you can edit it to create your own version.

Please submit the PDF and the \LaTeX{} versions of your re-write to the \textit{cs669-hw @ cs.pdx.edu} mailbox.  \textbf{Please also bring a paper copy to class}; this is the version that I will grade, unless one of us is sick!

\vspace{15pt}    % pt is the abbreviation for a point, 12 pt = 1 pc

\begin{center}
\large
\textbf{The Largest Prime}
\end{center}
\normalsize
\noindent
Suppose that there were a largest prime number $p$. 
Then consider the product $P = \prod_{i=1}^{p} \; i$. 
Then $P\,+\,1$ 
cannot be divided evenly by any of the numbers
up to $p$, $2,3,4,\ldots,p$ because each of these divides the left
factor evenly, but not the right factor, hence not their sum.
(Recall that if $a_1$ divides $a_2$ and $a_2=a_3+a_4$ then if $a_1$
divides $a_3$, it will also divide $a_4$.)  
Since we are assuming
$p$ is the largest prime,
$P\,+\,1$ can have no prime factors greater than $p$,
hence $P\,+\,1$ is a prime, and
it is greater than $p$, since $P \ge p$.
This contradicts the maximality of $p$.  
Hence the assumption that
$p$ is the largest prime must be false, and so there is no largest
prime.


\end{document}
