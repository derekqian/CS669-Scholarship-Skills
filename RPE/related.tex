\section{Related Work}
\label{sec:related}
There has been much research on building automatic tools for traceability link recovery (TLR) from software artifacts \cite{antoniol_recoveringtraceability_2002} \cite{deLucia_incremental_2006} \cite{marcus_recovering_2003}.
TLR technology scans the software artifacts including source code and documentation and recovers the semantic connections by information retrieval (IR) methods.
%\todo{improve}This technology has shown to be advanced and successful \cite{spanoudakis_software_2004}.
However, IR techniques cannot completely substitute the human decision-maker in the linking process.
Hayes et al. suggests such IR techniques should only be used to generate an appropriate list of candidate links which are then evaluated by software analysts \cite{hayes_advancing_2006}.
For this reason, these automatic techniques have not been widely adopted and most traceability links still need to be recovered manually.
Link acquisition remains human-intensive and with high initial cost as reported in case studies on industrial processes and traceability experiences \cite{lindvall_practical_1996} \cite{ramesh_implementing_1995} \cite{asuncion_an_2007} \cite{gotel_extended_1997} \cite{neumuller_case_2006}.

The evolution nature of software artifacts makes the situation even worse.
No matter how the traceability links are created -- automatically or manually,
these links are probably only valid for a certain version of the software.
As software evolves during the development and maintenance process, software artifacts changes constantly.
These changes might invalidate the existing traceability links.
For manually recovered links, we must reestablish these links manually again, which is arduous.
For automatically recovered links, we can re-run the TLR tool to reestablish them.
However, this solution is computationally costly for interactive use during the software development stage.
This intimidates people from using TLR tools in their daily work.
There has been research addressing this problem by improving existing IR based methods \cite{jiang_incremental_2008}.
Our approach differs in that we try to migrate valid links instead of re-establishing these links.
%However, these work didn't consider the situations specific to hardware related software.

%An intuitive thought is to use the semantic information,
%that is the intrinsic meaning of the code.
%However, we can't find a formal way of defining this information.
