% maintenace is important, takes time.
% reason: code size, spec size, relationship

% the lowest level HW/SW interface primitive is found in the form of SW programmable registers.
% There are other vital aspects of the HW/SW interface, such as interrupts but these are not covered within the scope of this paper.
% the software interface of an IP block as an address-mapped bus (from processor) being mapped into registers that provide the configuration, control and status interface of the hardware logic.
\section{Introduction}
\label{sec:introduction}

Significant portion of software engineering efforts is devoted to maintaining legacy software systems~\cite{abran_maintenance_1991}.
It is estimated that up to 60\% of software maintenance efforts is spent on program comprehension~\cite{lawrence_software_2006}.
Software developers often find it difficult to extract the knowledge about a software system,
because such knowledge is often implicitly expressed and distributed in a variety of software artifacts.
Examples of software artifacts include source code, design documents, application programming interface specifications, bug reports, etc.

To capture such knowledge explicitly, various techniques have been developed to recover the semantic connections among software artifacts,
especially between documentation and source code \cite{antoniol_recoveringtraceability_2002} \cite{deLucia_incremental_2006} \cite{marcus_recovering_2003}.
Such semantic connections are often called \textit{traceability links}.
Effectively maintaining the traceability links helps improving developer productivity and software quality \cite{neumuller_case_2006}.

Unfortunately, these techniques are not readily applicable to specification and implementation of hardware/software (HW/SW) interfaces,
even though such implementation causes most system failures in both Windows and Linux operating systems \cite{swift_os_2003} \cite{chou_os_2001},
HW/SW interface inter-connects hardware and software and defines the protocol for hardware and software to communicate.
The implementation of HW/SW interfaces typically include hardware designs and device drivers. There is also software implementation of hardware design, e.g., virtual devices in virtual machines such as QEMU~\cite{Bellard05}
%and virtual device prototypes, who implement the communication logic on both sides of the interfaces.
%The implements are designed for a variety of computer components, and form a large part of operating systems and virtual machines.

A HW/SW interface specification describes a HW/SW interface in natural language and guides its implementation.
The specification plays a crucial role in the comprehension and maintenance of the implementation.
The key reason why existing traceability link recovering techniques do not work for HW/SW interfaces is that these traceability links need to be fine-grained,
which are not supported by the existing techniques.
HW/SW interfaces are mainly composed of software-programmable hardware registers mapped to the processor address space.
The low-level abstraction of the HW/SW interface demands the fine-grained traceability links.
These traceability links need to connect statements which operate the registers or expressions which represent certain bits of registers to the register bit descriptions in the specification. However, previous techniques maintain traceability links between classes or functions of the source code and sections or paragraphes of the specification, which is to coarse-grained.

% Second, the evolution nature of software artifacts makes the situation even worse.
% No matter how the traceability links are created -- automatically or manually, these links are probably only valid for a certain version of the software.
% As software evolves during the development and maintenance process, software artifacts changes as well.
% These changes might invalidate the existing traceability links.
% For automatically recovered links, we can re-run the traceability recovery tool to reestablish them.
% However, this solution is computationally costly for interactive use during software development stage.
% For manually recovered links, we must reestablish these links manually again, which is arduous.
% This intimidate people from using these tools in their daily work.

% Second, previous traceability link recovering approaches use information retrieval techniques which have the assumption that all software artifacts are plain texts, and the key words in the texts can express the semantic information of the software artifacts precisely.
% These key words form the digital fingerprint of the software artifacts, and the digital fingerprints for the related software artifacts are highly similar, while the digital fingerprints for unrelated software artifacts are highly different.
% The assumption for IR techniques does not hold for device drivers and device specifications.
% People usually organize device specification by hardware modules and the registers inside the hardware modules.
% The semantic information is often encoded into binary numbers, and presented in tables.
% These modules and registers do not relate to the code in an explicit way.

We present a framework for managing traceability links between specification and implementation of HW/SW interfaces.
This framework supports fine-grained links between basic elements of specifications and constructs of implementation languages.
A simple, but effective data model is developed to support the framework and organize the data of the traceability links.
Our framework also supports automatic migration of valid traceability links as implementation evolves.

We have implemented a prototype tool, namely coDoc, in support of this framework.
With coDoc, developers can (1) import and manage implementation source code and interface specifications, (2) select a piece of implementation (POI) and a piece of specification (POS) of interest, (3) create a traceability link connecting the selected POI and POS, (4) check out a different version of source code, and (5) migrate, review, and verify existing traceability links. We applied coDoc to source code of three virtual devices from QEMU and their specifications, recovered traceability links, and migrated these links to newer versions of source code .
Experimental results show that our framework fully supports the types of traceability links between the virtual devices and their specifications,
maintain traceability links with required granularity, and accurately migrates valid links from version to version.
\todo{the data need to be updated}\hlc{It preserves 70\% of valid traceability links when source code evolves to a new version.}

This paper makes the following key contributions:
\begin{itemize}
\item Presented a framework for managing traceability links between specification and implementation of HW/SW interfaces;  
\item Developed a simple, but effective data model for capturing and recording links; 
\item Identified appropriate anchors in implementation and specification that support fine-granularity and migration of valid links as implementation evolves; 
\item Developed coDoc, a prototype tool implementing the approach and evaluated our approach on realistic examples.
\end{itemize}

The rest of the paper is organized as follows.
Section \ref{sec:problem} elaborates on the problem addressed in this paper.
Section \ref{sec:approach} and Section \ref{sec:implementation} illustrate the design and implementation details of our approach.
Section \ref{sec:evaluation} provides preliminary evaluation results.
Section \ref{sec:related} reviews related work.
We conclude this paper and discuss future work in Section \ref{sec:conclusion}.

